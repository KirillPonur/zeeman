Реально наблюдаемое в эксперименте число зеемановских линий существенным образом зависит от их относительных интенсивностей. Интенсивности зеемановских компонент могут быть рассчитаны из соображений симметрии. Результаты расчетов приведены в \ref{tab:1}.

В качестве примера приведем результаты расчета зеемановского спектра, соответствующего переходу 
$J\rightarrow J+1$
 между комбинирующими уровнями 
 $E_1(J_1=1,L_1=2,S_1=1)$ и $E_2(J_2=2,L_2=2,S_2=1)$,
  тогда согласно формуле \ref{eq:21} 
  $g_1=\frac12$, $g_2=\frac32$. 
  Переходы, на которых возможно получение зеемановских компонент, показаны стрелками на \ref{fig:3}, а в таблице \ref{tab:2} приведены поляризация и интенсивности соответствующих линий. 

% Как видно из \ref{eq:11} и таблиц \ref{tab:1}, \ref{tab:2}, зеемановский спектр зеркально симметричен относительно несмещенной линии, поэтому достаточно рассчитать половину спектра.
% ТАБЛИЦА 1
% РИСУНОК 3
% ТАБЛИЦА 2